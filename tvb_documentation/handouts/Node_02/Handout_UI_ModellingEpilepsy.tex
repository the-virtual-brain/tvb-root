%!TEX TS-program = pdflatex                                          %
%!TEX encoding = UTF8                                                %
%!TEX spellcheck = en-US                                             %
%
%%%%%%%%%%%%%%%%%%%%%%%%%%%%%%%%%%%%%%%%%%%%%%%%%%%%%%%%%%%%%%%%%%%%%%
% Handout_ModelingStructuralLesions.tex
% A handout to follow during the hands-on sessions 
% 
% Authors: 
% 2014-06-XX : Timothée Proix - Design project and wrote handout
% 2014-11-XX : Paula Sanz-Leon - Update text and figures to macthc TVB 1.3 functionalities.
% 
%%%%%%%%%%%%%%%%%%%%%%%%%%%%%%%%%%%%%%%%%%%%%%%%%%%%%%%%%%%%%%%%%%%%%%
% based on the tufte-latex template                                  %

\documentclass{tufte-handout}
%\geometry{showframe}% for debugging purposes -- displays the margins
\usepackage{amsmath}

% Set up the images/graphics \underline{\textbf{Analysis}}
\usepackage[pdftex]{graphicx}
\setkeys{Gin}{width=\linewidth,totalheight=\textheight,keepaspectratio}
\graphicspath{{../figures/} {../../framework_tvb/tvb/interfaces/web/static/style/img/} {../../framework_tvb/tvb/interfaces/web/static/style/img/nav/}}

\title{The Virtual Brain: Hands on Session \#2}
\date{14th November 2014} 

% The following package makes prettier tables.  
\usepackage{booktabs}

% The units package provides nice, non-stacked fractions and better spacing
% for units.
\usepackage{units}
\usepackage[svgnames]{xcolor}

% The fancyvrb package lets us customize the formatting of verbatim
% environments.  We use a slightly smaller font.
\usepackage{fancyvrb}
\fvset{fontsize=\normalsize}

% Small sections of multiple columns
\usepackage{multicol}

% For adjustwidth environment
\usepackage[strict]{changepage}

% For formal definitions
\usepackage{framed}

% And some maths
\usepackage{amsmath}  % extended mathematics

% Resume a list
\usepackage{enumitem}

% Background image

\usepackage{wallpaper}

% Provides paragraphs of dummy text
\usepackage{lipsum}

% These commands are used to pretty-print LaTeX commands
\newcommand{\doccmd}[1]{\texttt{\textbackslash#1}}% command name -- adds backslash automatically
\newcommand{\docopt}[1]{\ensuremath{\langle}\textrm{\textit{#1}}\ensuremath{\rangle}}% optional command argument
\newcommand{\docarg}[1]{\textrm{\textit{#1}}}% (required) command argument
\newenvironment{docspec}{\begin{quote}\noindent}{\end{quote}}% command specification environment
\newcommand{\docenv}[1]{\textsf{#1}}% environment name
\newcommand{\docpkg}[1]{\texttt{#1}}% package name
\newcommand{\doccls}[1]{\texttt{#1}}% document class name
\newcommand{\docclsopt}[1]{\texttt{#1}}% document class option name

\newcommand\blfootnote[1]{\begingroup
         \renewcommand\thefootnote{}\footnote{\phantom{\thefootnote} #1}%
         \addtocounter{footnote}{-1}%
         \endgroup
          }

% Colours: environment derived from framed.sty: see leftbar environment definition
\definecolor{formalshade}{rgb}{0.95,0.95,1}
\definecolor{simulationshade}{rgb}{0.92, 1.0, 0.95}

% Title rule
\newcommand{\HRule}{\rule{\linewidth}{0.5mm}}

% Framed  coloured boxes

%% Blue box: for steps regarding analysis and such
\newenvironment{formal}{%
  \def\FrameCommand{%
    \hspace{1pt}%
    {\color{DarkBlue}\vrule width 2pt}%
    {\color{formalshade}\vrule width 4pt}%
    \colorbox{formalshade}%
  }%
  \MakeFramed{\advance\hsize-\width\FrameRestore}%
  \noindent\hspace{-4.55pt}% disable indenting first paragraph
  \begin{adjustwidth}{}{7pt}%
  \vspace{2pt}\vspace{2pt}%
}
{%
  \vspace{2pt}\end{adjustwidth}\endMakeFramed%
}

%% Green box: for steps regarding simulatio **only**
\newenvironment{simulation}{%
  \def\FrameCommand{%
    \hspace{1pt}%
    {\color{ForestGreen}\vrule width 2pt}%
    {\color{simulationshade}\vrule width 4pt}%
    \colorbox{simulationshade}%
  }%
  \MakeFramed{\advance\hsize-\width\FrameRestore}%
  \noindent\hspace{-4.55pt}% disable indenting first paragraph
  \begin{adjustwidth}{}{7pt}%
  \vspace{2pt}\vspace{2pt}%
}
{%
  \vspace{2pt}\end{adjustwidth}\endMakeFramed%
}

%% Orange box: for verbose descriptions
\newenvironment{blah}{%
  \def\FrameCommand{%
    \hspace{1pt}%
    {\color{DarkOrange}\vrule width 2pt}%
    {\color{PeachPuff}\vrule width 4pt}%
    \colorbox{PeachPuff}%
  }%
  \MakeFramed{\advance\hsize-\width\FrameRestore}%
  \noindent\hspace{-4.55pt}% disable indenting first paragraph
  \begin{adjustwidth}{}{7pt}%
  \vspace{2pt}\vspace{2pt}%
}
{%
  \vspace{2pt}\end{adjustwidth}\endMakeFramed%
}

%%%%%%%%%%%%%%%%%%%%%%%%%%%%%%%%%%%%%%%%%%%%%%%%%%%%%%%%%%%%%%%%%%%%%%%%%%%%%%
%                      The document starts here                              %
%%%%%%%%%%%%%%%%%%%%%%%%%%%%%%%%%%%%%%%%%%%%%%%%%%%%%%%%%%%%%%%%%%%%%%%%%%%%%%
\begin{document}
\thispagestyle{plain}
\LLCornerWallPaper{1.5}{background.png}
\begin{titlepage}
\begin{center}
% Upper part of the page. The '~' is needed because \\
% only works if a paragraph has started.
\includegraphics[width=1.5\textwidth]{./tvb_logo_transparent_square.png}~\\[0.5cm]

% Title
\begin{fullwidth}
\HRule \\[0.2cm]
\begin{center}
{ \huge \bfseries Hands-on Session \#2 \\ [0.2cm] Modelling Epilepsy \\[0.1cm] }
{ \large \bfseries November 14, 2014 \\[0.2cm]}
\end{center}
\HRule \\[0.2cm]
\end{fullwidth}

\end{center}
\end{titlepage}

\newpage
\ClearWallPaper


\begin{abstract}
\noindent 
It is possible to use TVB to model a specific subject such as an epileptic
patient. Using relevant neural mass models, TVB allows to ask multiple
questions such as the localisation of the epileptogenic zone or the validity
of different neuroimaging modalities to assess the epileptogenicity of a brain
structure. Here we will present a example of such a modelisation.
\end{abstract}

%\begin{fullwidth} % uncomment this environment to get full texwidth paragraphs 
 
%\end{fullwidth}

\section{Objectives}\label{sec:objectives}
\newthought{The main goal} 
of this session is to provide a clear understanding of how we can reproduce
clinically relevant scenarios such as stimulation, modelisation of forward
solution of sEEG and EEG during a seizure; and, modelisation of surgery with
resection of a part of the brain. Make sure you are working with at least TVB
1.3. If you are using version 1.2.X or older versions, there might be some
features that are not available you experiencie difficulties importing the
projects we have just shared with you.

\begin{blah}
Another important step is to process data from neuroimaging modalities into
TVB. \href{There is a pipeline written by Timothée
Proix}{https://github.com/timpx/scripts} that let you easily prepare your data
in a format suitable for import into TVB. All you will need for this pipeline
will be a T1 MRI and a Diffusion MRI from the subject. You can find these data
in databases such as the Human Connectome project. At the end of the pipeline,
you will have a connectivity, a surface and a region mapping, i.e. you will be
able to do region based and surface based simulations. We will use the
different functionalities that you have learned in the previous sessions.
\end{blah}

\subsection{Project II: Modelling Epilepsy}\label{sec:project_data}

\newthought{In this project}, 
all the data were already generated. We'll only go through the necessary steps
required to reproduce the simulations listed in Table~\ref{tab:simtab}, along
with the relevant outline. You can always start over, click along and/or try
to change parameters. We will use the default subject connectivity matrix and
surface.


\begin{margintable}
  \centering
  \fontfamily{ppl}\selectfont
  \begin{tabular}{l}
    \toprule
    Name \\
    \midrule
    Exploring the Epileptor model\\
    \\
    Region based simulation of an epileptic patient \\
    $\quad$\textit{Region\_TemporalLobe} \\
    $\quad$\textit{Region\_TemporalLobe\_sEEG\_EEG}  \\
    \\
    Surface based simulation of an epileptic patient \\
    $\quad$\textit{Surface\_TemporalLobe\_sEEG\_EEG}  \\
    \\
    Applying a stimulus to trigger seizures \\
    $\quad$\textit{Surface\_Stimulation}  \\ 
    \\
    Modeling surgical resection\\
    $\quad$\textit{Surface\_Resection} \\
    \bottomrule
  \end{tabular}
  \caption{Simulations in this project.}
  \label{tab:simtab}
\end{margintable}

% let's start a new thought -- a new section

\subsection{Exploring the Epileptor model}\label{sec:epileptor}


\newthought{Before launching} 
any simulations, we will have a look at the phase space of the Epileptor model
to understand better its dynamics. We will use the phase plane interactive tool that you
have seen in probably in Session \#1.

\begin{formal}
  \begin{enumerate}[resume]
  \item Go to \textsc{simulator} $\rightarrow$ \textsc{Phase plane} and select the $\mathbf{Epileptor}$ model.
  \item Look at the phase space (Fig. \ref{fig:phase_space}). We have here the first population (variables $y_0$ in abscissa and 
  $y_1$ in ordinate).
  The left most intersection of the nullcline defines a stable fixed point whereas the rightmost intersection 
  is the center of a limit cycle. Both states are separated by a separatrix, as you can se by drawing different trajectories
  in this phase space (left click on the figure).
  \item You can also look at other variables in the phase space, such as $y_2$/$y_0$ (slow-fast subsystem) or 
  $y_5$/$y_4$ (second population), and change parameters to see what is the effect on the nullclines.
  \end{enumerate}
\end{formal}


\begin{figure}[h]
  \includegraphics[width=\linewidth]{Handout_UI_ModellingEpilepsy_PhaseSpace}%
  \caption{$y_0-y_1$ phase plane of the first population.}%
  \label{fig:phase_space}%
\end{figure}
\subsection{Region based simulation of an epileptic patient}

\newthought{We will model} 
a patient with temporal lobe epilepsy (TLE). Here, using the tools in
\underline{Set up Region model} we will set different values of
epileptogenicity ($x_0$ parameter in the Epileptor) according to the region
positions, thereby introducing heterogeneities in the network parameters. We
set the right limbic areas  (right hippocampus (rHC), parahippocampus (rPHC)
and amygdala (rAMYG)) as epileptic zones. We also add two lesser epileptogenic
regions: the superior temporal cortex (rTCS) and the ventral temporal cortex
(rTCV). 

%% Steps:
%1) Go to phase plane, save two par configs. One epileptic one non epileptic. 
%2) Got to set up region model. Set the all the nodes to non epileptic
%3) Set the limbic regions to epileptic. 
%4) Select all the nodes. Submit.

\begin{simulation}
  \begin{enumerate}
  \item In other words, in \underline{Set up Region model} assign to all the nodes the \underline{Dynamics} for which $\mathbf{x_0}$ has a value of value of $\mathbf{-2.2}$. Apply the non-epileptogenic configuration ($\mathbf{x_0}=\mathbf{-1.6}$) to the \underline{RigtLimbicAreas} selection. Select all the nodes and click on \underline{Submit Region Parameters} values.
  All the other model parameters are given in Table \ref{tab:modeltab}.

  \item Once back in the \textsc{Simulator} playground \underline{Configure the Visualizers} and select the \underline{Brain Viewer}. 
  \underline{Save your choices}.
  \item Set the $\mathbf{integration\:step\:size}$ to \textbf{\unit[0.1]{ms}}, select the $\mathbf{Temporal\:average\:monitor}$ with a \textbf{sampling period} of \textbf{\unit[1]{ms}} and set \textbf{\unit[6000]{ms}} as the simulation length. 
 \end{enumerate}
\end{simulation}

\begin{margintable}
  \centering
  \fontfamily{ppl}\selectfont
  \begin{tabular}{ll}
    \toprule
    Model parameter & Value \\
    \midrule
             $Iext$           &   3.1     \\
             $Iext2$          &   0.45    \\
             $R$              &   0.00035 \\
             $slope$          &   0.0     \\
    \bottomrule
  \end{tabular}
  \caption{Parameters for the Epileptor model}
  \label{tab:modeltab}
\end{margintable}

\begin{marginfigure}
  \includegraphics[width=\linewidth]{Handout_UI_ModellingEpilepsy_BrainMenuScaling}%
  \caption{Brain menu: you can increase the scaling of the signals.}%
  \label{fig:brain_menu}%
\end{marginfigure}

The results are already computed for you in \textit{Region\_TemporalLobe}


\begin{simulation}
  \begin{enumerate}
  \setcounter{enumi}{4}
  \item Visualize the time series. Click on \underline{Select Input Signals} and select all the regions. 
  From this same menu you can select which state variables of interest will be displayed. For instance, visualize 
  $y_0$ (Fig. \ref{fig:first_pop}) and then $y_3$ state (Fig. \ref{fig:second_pop}). 
	You will need to increase the scaling by clicking on \includegraphics[width=0.08\textwidth]{butt_brain_menu}(Fig. \ref{fig:brain_menu}). You can see a succession of 3 seizures, use the mouse
	 to zoom in and out in the time series area.
  \item Go back to the \textsc{Simulator} page and visualize the results in the \underline{Brain Viewer}, you will need to increase the rendering speed 
	(timesteps per Frame) by clicking on \includegraphics[width=0.08\textwidth]{butt_brain_menu}.
\end{enumerate}
\end{simulation}
  
\begin{figure}[h]
  \includegraphics[width=\linewidth]{Handout_UI_ModellingEpilepsy_TSPop1}%
  \caption{Time series for the first population}%
  \label{fig:first_pop}%
\end{figure}



\begin{figure}[h]
  \includegraphics[width=\linewidth]{Handout_UI_ModellingEpilepsy_TSPop2}%
  \caption{Time series for the second population}%
  \label{fig:second_pop}%
\end{figure}

\begin{blah}
  \begin{enumerate}[resume]
  The length of seizures here is not realistic ($\sim$\unit[2]{s}), but you can always obtain realistic time by multiplying
  all the derivatives of the model by a small factor.
  \end{enumerate}
\end{blah}
We now are going to run this simulation again, but with intracranial electrodes (sEEG) and EEG monitors.

\begin{simulation}
  \begin{enumerate}
  \item Copy the former simulation.
  \item Unselect the Temporal average monitor and select two new monitors (EEG and sEEG) with a \textbf{sampling period} of \textbf{\unit[1]{ms}} by hitting *Ctrl* (*Cmd* for Mac users) and at the same time clicking on the monitor you want to add.
  \item For this simulation, we will not use the \underline{Brain Viewer} so reconfigure the \underline{View} tabs accordingly.
\end{enumerate}
\end{simulation}

The results are already computed for you in \textit{Region\_TemporalLobe\_sEEG\_EEG}


\begin{simulation}
  \begin{enumerate}
    \setcounter{enumi}{3}
  \item Click on \underline{Results}.
  \item Click on the EEG time series and visualize them with the 3d/2d visualizer (Fig. \ref{fig:EEG}).
  \item Go back, click on the sEEG time series and visualize them with the 3d/2d visualizer (Fig. \ref{fig:sEEG}).
\end{enumerate}
\end{simulation}



\begin{figure}[h]
  \includegraphics[width=\linewidth]{Handout_UI_ModellingEpilepsy_EEG3d2d}%
  \caption{EEG 3d/2d visualizer}%
  \label{fig:EEG}%
\end{figure}

\begin{figure}[h]
  \includegraphics[width=\linewidth]{Handout_UI_ModellingEpilepsy_sEEG3d2d}%
  \caption{sEEG 3d/2d visualizer}%
  \label{fig:sEEG}%
\end{figure} 

\subsection{Surface based simulation of an epileptic patient}

 \newthought{To account} also for seizure propagation and not only seizure recruitment, we have to use surface-based simulations.
 This type of simulations allow for a more accurate representation of EEG/sEEG signals.
 
  \begin{simulation}
  \begin{enumerate}
  \item Copy the former simulation.
  \item Add a \underline{Brain Viewer} visualizer.
  \item Choose the TVB's $\mathbf{default\:surface}$, the corresponding  $\mathbf{local\:connectivity}$ and a $\mathbf{Local\:coupling\:strength}$ of $\mathbf{a=0.2}$.
  \item Add a $\mathbf{Spatial\:average}$ monitor with a \textbf{sampling period} of \textbf{\unit[1]{ms}}
  \item Click on \underline{Set up surface model} (Fig. \ref{fig:set_up_surface_parameters}). Choose working parameters
  ($\mathbf{Model\:parameter\:x_0}$, $\mathbf{Equation:\:Gaussian}$, $\mathbf{amp=0.6}$, $\mathbf{sigma=10.}$, $\mathbf{offset=-2.2}$) and 
  click on \underline{Apply equation}.
  Click on a location on the brain where you want the parameter setting to apply, then click on \underline{Add focal point}.
  Then click on \underline{Submit Surface parameters}. You can see that the field of the $x_0$ parameter is updated according to the selected equation.
  \end{enumerate}
\end{simulation}

\begin{figure}[h]
  \includegraphics[width=\linewidth]{Handout_UI_ModellingAnEpilepticPatient_SetUpSurfaceParameters}%
  \caption{Set up the surface parameters}%
  \label{fig:set_up_surface_parameters}%
\end{figure}

  The results are already computed for you in \textit{Surface\_TemporalLobe\_sEEG\_EEG}
 \begin{simulation}
  \begin{enumerate}
     \setcounter{enumi}{5}
  \item Click on \underline{Results}, click on the TimeSeriesSurface an visualize them in the 
  \underline{Brain Activity Visualizer} (Fig. \ref{fig:bv_surf}). You can use the arrows of the numpad to rapidly 
  move the brain.
  \item Go back,  click on \underline{TimeSeries} and visualize them with the \underline{Time Series Visualizer}. Change the scaling, change the selected channels and zoom in to see
  the seizure (Fig. \ref{fig:ts_surf}).
  \item Go back, click on \underline{TimeSeriesSEEG} and visualize them with the 3d/2d visualizer.
  Note the difference with the former region-based simulation.
  \item Go back, click on the EEG time series and visualize them with the 3d/2d visualizer.
  Note the difference with the region-based simulation.
\end{enumerate}
\end{simulation}

\begin{figure}[h]
  \includegraphics[width=\linewidth]{Handout_UI_ModellingEpilepsy_TemporalAverageTimeSeriesSurface}%
  \caption{Temporally averaged time series for a surface simulation}%
  \label{fig:bv_surf}%
\end{figure}

\begin{figure}[h]
  \includegraphics[width=\linewidth]{Handout_UI_ModellingEpilepsy_SpatialAverageTimeSeries}%
  \caption{Spatial averaged time series for a surface simulation}%
  \label{fig:ts_surf}%
\end{figure}

% \begin{figure}[h]
%   \includegraphics[width=\linewidth]{Handout_UI_ModellingAnEpilepticPatient_sEEGSurface}%
%   \caption{sEEG for a surface simulation}%
%   \label{fig:surf_sEEG}%
% \end{figure}

% \begin{figure}[h]
%   \includegraphics[width=\linewidth]{Handout_UI_ModellingAnEpilepticPatient_EEGSurface}%
%   \caption{EEG for a surface simulation}%
%   \label{fig:surf_EEG}%
% \end{figure}

\subsection{Applying A Stimulus To Trigger Seizures}

\newthought{Now} we are going to simulate a stimulation.
We set the whole brain to non-epileptogenic but close to the threshold

\begin{margintable}
  \centering
  \fontfamily{ppl}\selectfont
  \begin{tabular}{ll}
    \toprule
    Space stimulation parameters & Value \\
    \midrule
             $amp$              &   1.0  \\
             $radius$           &   5.0   \\
             $sigma$            &   1.0  \\
             $offset$           &   0.0  \\
    \midrule
    \midrule
    Time stimulation parameters & Value \\
    \midrule
             $onset$          &   2000.0  \\
             $tau$            &   20.0    \\
             $T$              &   4000.0  \\
             $amp$            &   10.0    \\
    \bottomrule
  \end{tabular}
  \caption{Space and time parameters for the stimulus}
  \label{tab:stimtab}
\end{margintable}

  \begin{formal}
  \begin{enumerate}
  \item Go to \textsc{stimulus} $\rightarrow$ \textsc{Surface Stimulus}
  \item Give a name to the new stimulus
  \item Choose a \underline{Sigmoid} stimulation in space with the parameters given in Table \ref{tab:stimtab}
  \item Choose a \underline{PulseTrain} stimulation in time with parameters given in Table \ref{tab:stimtab} (Fig. \ref{fig:stim_st})
  \item Click on \underline{Edit Focal Points and View}
  \item Choose a focal point (Fig. \ref{fig:stim_foc})
  \item \underline{Save the new stimulus on surface}
  \end{enumerate}
\end{formal}


\begin{figure}[h]
  \includegraphics[width=\linewidth]{Handout_UI_ModellingEpilepsy_SpatialAverage}%
  \caption{Spatio temporal pattern of the stimulus}%
  \label{fig:stim_st}%
\end{figure}

\begin{figure}[h]
  \includegraphics[width=\linewidth]{Handout_UI_ModellingAnEpilepticPatient_StimulationFocalPoint}%
  \caption{Focal point for a surface stimulation}%
  \label{fig:stim_foc}%
\end{figure}

The stimulus was already set for you under the name \textit{Surface\_SquareStimulus}

  \begin{simulation}
  \begin{enumerate}
  \setcounter{enumi}{7}
  \item Go to \textsc{simulator} and copy the former simulation.
  \item Choose the \textit{Surface\_SquareStimulus} stimulus.
  \item Set the parameter $\mathbf{x_0}$ to $\mathbf{-2.1}$
  \item Choose only a \textbf{Spatial average} monitor.
  \item Set the $\mathbf{Simulation\:Length}$ to \textbf{\unit[4000]{ms}}.
 
\end{enumerate}
\end{simulation}


You can see the result of this simulation in \textit{surface stimulation}

\begin{marginfigure}
  \includegraphics[width=\linewidth]{Handout_UI_ModellingAnEpilepticPatient_ChooseChannelsStimulation}%
  \caption{Signals Input menu: you can choose the sources of interest.}%
  \label{fig:choose_channels}%
\end{marginfigure}


  \begin{simulation}
  \begin{enumerate}
  \setcounter{enumi}{12}
  \item Click on \underline{Results}, click on the TimeSeries and visualize the Surface Average Time Series with 
  the \underline{Animated Time Series} visualizer.
  \item \underline{Select channel} \#67 to \#73 (Fig. \ref{fig:choose_channels}). Increase the page size in \includegraphics[width=0.08\textwidth]{butt_brain_menu} to include time \textbf{\unit[2000]{ms}}
  (Fig. \ref{fig:bm_stim}) and better see the effect of the stimulation (Fig. \ref{fig:stim_ts}).
 
\end{enumerate}
\end{simulation}


\begin{figure}[h]
  \includegraphics[width=\linewidth]{Handout_UI_ModellingAnEpilepticPatient_StimulationTimeSeries}%
  \caption{Time Series for a stimulation}%
  \label{fig:stim_ts}%
\end{figure}

\begin{marginfigure}
  \includegraphics[width=\linewidth]{Handout_UI_ModellingAnEpilepticPatient_BrainMenuStimulation}%
  \caption{Brain menu: Increase the scaling to see the time \unit{2000}[ms].}%
  \label{fig:bm_stim}%
\end{marginfigure}


\subsection{Modeling surgical resection}

\newthought{Surgical resection} 
is used for around 20\% of epileptic patient whose seizures are drug-
resistant. We can imagine in this case that a small part of the brain of the
patients  has been resected ( for instance the hippocampus, the amygdala and
the parahippocampal cortex) and  we are going to simulate that.

\begin{simulation}
  \begin{enumerate}
  \item Go to \textsc{connectivity} $\rightarrow$ \textsc{Large scale Connectivity}.
  \item Select the nodes rAMYG, rHC and rPHC via the \textit{RightLimbicAreas} selection, delete all their in-out and out-in connections
  with the other nodes, give a name to the selection and save it. (Fig. \ref{fig:resec})
  \item As we choose the same regional parameters that in the first simulation, we are going to copy them directly from our
  former setting. Go to the \textit{Region\_TemporalLobe} simulation, select and copy the values for the parameter  $x_0$. 
  \item Copy the surface simulation \textit{Surface\_TemporalLobe\_sEEG\_EEG} and paste the array of values in the parameter $\mathbf{x_0}$.
  With this direct copy-paste method, we avoid the step previously done via the \underline{Set Up Region Model} panel.
  You can also use you own vector, as long as the length of this vector correspond to the number of regions.
  Change the values $\mathbf{-1.6}$ by $\mathbf{-2.2}$ in this array (i.e. we replace the dynamics of the resected node by a stable node).
  \item Choose only a \textbf{Spatial average} monitor with a \textbf{sampling period} of \textbf{\unit[1]{ms}}.
  \item Remove the \underline{Brain Viewer} visualizer.
  \item Do not forget to choose the right connectivity matrix and you are ready to launch the simulation.
  \end{enumerate}
\end{simulation}

The results are given in \textit{Surface\_Resection}.

\begin{figure}[h]
  \includegraphics[width=\linewidth]{Handout_UI_ModellingAnEpilepticPatient_ConnectivityMatrixResection}%
  \caption{Focal point for a surface stimulation}%
  \label{fig:resec}%
\end{figure}

\begin{simulation}
  \begin{enumerate}
    \setcounter{enumi}{5}
  \item Click on \underline{Results}, then TimeSeries and visualize the spatial average time series with the a Time Series visualizer.
  Don't forget to increase the \underline{Scaling} and \underline{Select all} channels. (Fig. \ref{fig:ts_resec})
  \end{enumerate}
\end{simulation}

\begin{figure}[h]
  \includegraphics[width=\linewidth]{Handout_UI_ModellingAnEpilepticPatient_TimeSeriesResection}%
  \caption{Time Series after a resection}%
  \label{fig:ts_resec}%
\end{figure}


\section{More Documentation}\label{sec:more-doc}

For more documentation on The Virtual Brain platform, please see \citet{Sanz-
Leon_2013, Woodman_2014} for technical details; and \citet{Jirsa_2014} for
information about the \textbf{Epileptor} model.


\section{Support}\label{sec:support}

The official TVB webiste is \url{www.thevirtualbrain.org}.  
All the documentation and tutorials are hosted on \url{the-virtual-brain.github.io}.
You'll find our public \smallcaps{git} repository at \url{https://github.com/the-virtual-brain}. 
For questions and bug reports we have a users group \url{https://groups.google.com/forum/#!forum/tvb-users}

\subsection{License}

This tutorial is Licensed under the Creative Commons Attribution-NonCommercial-ShareAlike 3.0 Unported
License (the ``License''). You may not use this file except in compliance with
the License. You may obtain a copy of the License at \url{https://creativecommons.org/licenses/by-nc-sa/3.0/}. 
Unless required by applicable law or agreed to in writing, software distributed under the License
is distributed on an \textsc{``as is'' basis, without warranties or conditions
of any kind}, either express or implied. See the License for the specific
language governing permissions and limitations under the License.
Copyright Timothée Proix and Paula Sanz-Leon 2014.

\bibliography{tvb_references}
\bibliographystyle{plainnat}

\end{document}